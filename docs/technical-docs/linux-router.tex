\subsection{Configure your Linux Computer as an AP of the
project}\label{configure-your-linux-computer-as-an-ap-of-the-project}

Please watch the figure on assets/network-diagram.png for understanding
this article.

\subsubsection{Hardware you need:}\label{hardware-you-need}

\begin{itemize}
\item
  A wifi adapter network that supports AP virtual interface (we used an
  Alfa AWUS036ACH-C, but some integrated wifi cards can handle
  promiscuous mode and AP virtual interfaces)
\item
  A computer running Linux with systemd.
\end{itemize}

\subsubsection{Software you need:}\label{software-you-need}

\begin{itemize}
\item
  \href{https://github.com/lakinduakash/linux-wifi-hotspot}{Linux Wifi
  Hotspot}
\end{itemize}

\subsubsection{Brief overview of the
steps:}\label{brief-overview-of-the-steps}

\begin{itemize}
\item
  Patch the create\_ap program that comes with Linux Wifi Hotspot for
  enabling an static IP for the raspberry pi.
\item
  Change the Internet interface and the Wifi interface on the ap.conf.
\item
  Execute!
\end{itemize}

\paragraph{Patch the create\_ap
program}\label{patch-the-create_ap-program}

By default,
\href{https://github.com/lakinduakash/linux-wifi-hotspot}{Linux Wifi
Hotspot} executes: -
\href{https://wiki.gentoo.org/wiki/Hostapd}{hostapd}: For creating the
wifi access point -
\href{https://wiki.archlinux.org/title/Dnsmasq}{dnsmasq}: For adding
DHCP server to the network. -
\href{https://wiki.archlinux.org/title/Iptables}{iptables}: For
forwarding all the traffic that comes from the AP network (in other
words, in your AP network your computer acts as a router, forwarding the
packets).

Using the \textbf{create\_ap} script that comes with the Linux Wifi
Hotspot packages become very handy, since we don't have to create the
hostapd, dnsmasq and iptables ourselves, buuut has a catch: \textbf{it
doesn't allow us to put dhcp rules (such as static ip for the raspberry
pi) on his config file}.

This is because the \textbf{create\_ap} script automatically generates
the configuration for hostapd, dnsmasq and iptables given the
configuration file you gave it (and obviously this config file doesn't
have an option for adding dhcp rules\ldots),
\href{https://github.com/lakinduakash/linux-wifi-hotspot/blob/d73242ab812284b5ed65275f630b8bc306b725c5/src/scripts/create_ap\#L1762}{you
can watch the code that generetes the dnsmasq rules here}.

So, we had to patch the \textbf{create\_ap} script for adding our custom
rule! Yay! First of all find where is the \textbf{create\_ap} script
living on your computer:

\begin{verbatim}
$ whereis create_ap
# /usr/bin/create_ap if you're using archlinux
\end{verbatim}

Then, patch it with:

\begin{verbatim}
$ patch /usr/bin/create_ap < create_ap_raspberry_pi.patch
\end{verbatim}

Be aware that this will add a custom rule for the dnsmasq program, which
is:

\begin{verbatim}
dhcp-host=e4:5f:01:f2:b4:df,192.168.33.213
\end{verbatim}

If you don't have the same raspberry we're using, the raspberry mac
address will change, this means you will have to tweak this patch file.
If you're following our network diagram you don't have to
change the ip address.

\paragraph{Change the interfaces of the config
file}\label{change-the-interfaces-of-the-config-file}

Every linux distribution names his network devices differently (and
everyone has different devices), so execute the following command:

\begin{verbatim}
$ ip a
....
17: wlp0s20f0u2: <BROADCAST,MULTICAST,UP,LOWER_UP> mtu 2312 qdisc mq state UP 
18: enp0s20f0u6: <BROADCAST,MULTICAST,UP,LOWER_UP> mtu 1500 qdisc fq_codel state UP 

\end{verbatim}

In my case I see that the Alfa Wifi Card is named wlp0s20f0u2, and the
ethernet connection to the internet is named enp0s20f0u6, so change the
ap.conf acording to this interface names:

\begin{verbatim}
....
WIFI_IFACE=wlp0s20f0u2
INTERNET_IFACE=enp0s20f0u6
\end{verbatim}

\paragraph{Run the access point!}\label{run-the-access-point}

Everything is ready now! Execute:

\begin{verbatim}
$ sudo create_ap --config ap.conf
\end{verbatim}

If any problem occurs is 99\% problem of your wifi card. See the
requirements on the hardware section.
