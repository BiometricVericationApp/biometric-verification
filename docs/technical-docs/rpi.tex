\subsection{Raspberry Pi 4
Configuration}\label{raspberry-pi-4-configuration}

We use the raspberry pi 4 as a: - MQTT Broker - EdgeImpulse node for
running the IA for detecting objects.

\subsubsection{Configure the raspberry
pi}\label{configure-the-raspberry-pi}

You can configure your raspberry pi liunx image with
\href{https://github.com/raspberrypi/rpi-imager}{rpi imager}, which
comes pretty handy since you can:

\begin{itemize}
\tightlist
\item
  Enable wifi with an SSID and password.
\item
  Enable ssh and give it a public key that is allowed to connect to the
  rpi
\item
  Add a username and password
\end{itemize}

This saved us a lot of time, and we didn't have to use any ethernet
cable or screen for configuring it.

\subsubsection{Install mosquitto broker}\label{install-mosquitto-broker}

Install mosquitto with:

\begin{Shaded}
\begin{Highlighting}[]
\CommentTok{\# We\textquotesingle{}re using the raspbian image for the rpi}
\ExtensionTok{$}\NormalTok{ sudo apt{-}get update }\KeywordTok{\&\&} \FunctionTok{sudo}\NormalTok{ apt{-}get upgrade}
\ExtensionTok{$}\NormalTok{ sudo apt{-}get install mosquitto}
\end{Highlighting}
\end{Shaded}

\subsubsection{Run mosquitto broker}\label{run-mosquitto-broker}

Once mosquitto is installed, copy our configuration file and run it
with:

\begin{Shaded}
\begin{Highlighting}[]
\ExtensionTok{$}\NormalTok{ mosquitto }\AttributeTok{{-}c}\NormalTok{ mosquitto.conf }\AttributeTok{{-}v}
\end{Highlighting}
\end{Shaded}

It will serve the mqtt server on port 1883.

\subsubsection{}\label{section}
