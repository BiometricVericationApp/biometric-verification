\subsection{Raspberry Pi 4 Configuration}\label{raspberry-pi-4-configuration}

We use the raspberry pi 4 as a: - MQTT Broker - EdgeImpulse node for
running the IA for detecting objects.

\subsubsection{Configure the raspberry pi}\label{configure-the-raspberry-pi}

You can configure your raspberry pi liunx image with
\href{https://github.com/raspberrypi/rpi-imager}{rpi imager}, which
comes pretty handy since you can:

\begin{itemize}
\item
  Enable wifi with an SSID and password.
\item
  Enable ssh and give it a public key that is allowed to connect to the
  rpi
\item
  Add a username and password
\end{itemize}

This saved us a lot of time, and we didn't have to use any ethernet
cable or screen for configuring it.

\subsubsection{Install mosquitto broker}\label{install-mosquitto-broker}

Install mosquitto with:

\begin{verbatim}
# We're using the raspbian lite image for the rpi
$ sudo apt-get update && sudo apt-get upgrade
$ sudo apt-get install mosquitto
\end{verbatim}

\subsubsection{Run mosquitto broker}\label{run-mosquitto-broker}

Once mosquitto is installed, copy our configuration file and run it
with:

\begin{verbatim}
$ mosquitto -c mosquitto.conf -v
\end{verbatim}

It will serve the mqtt server on port 1883.


This doesn't make the server boot on startup, so we created a system.d service that does just that! Enable it with:
\begin{verbatim}
cp mosquitto.service /etc/systemd/system/ 
sudo systemctl enable mosquitto.service
sudo systemctl start mosquitto.service
\end{verbatim}

\subsubsection{Run authenticate ML model}\label{run-ia}
Download the model code, and be sure that you have a version of python3 installed.

Then, enter the source folder and execute:
\begin{verbatim}
$ pip install -r requirements.txt
$ python3 main.py
\end{verbatim}

This doesn't start the program at start, so we created a system.d service that does just that!
\begin{verbatim}
cp predict.service /etc/systemd/system/ 
sudo systemctl enable predict.service
sudo systemctl start predict.service
\end{verbatim}
